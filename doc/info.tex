\documentclass{article}
\usepackage{amsmath}
\usepackage{hyperref}

% Margins
\topmargin=-0.45in
\evensidemargin=0in
\oddsidemargin=0in
\textwidth=6.5in
\textheight=9.0in
\headsep=0.25in

\title{Game Structure Overview}
\author{}
\date{}


\begin{document}

\maketitle

\section{Moteur de jeu}
Le \textbf{Moteur de jeu} est le composant centrale qui gére le boucle de jeu et coordone les composant de la structure MVC, Modéle, View, Controlleur.

\begin{itemize}
    \item Boucle de jeu, et son arret.
    \item Crée composant MVC et les lie entre eux.
\end{itemize}

\begin{center}
    \begin{minipage}[c]{0.5\textwidth}
        \begin{verbatim}
class GameEngine
    render = new Renderer()
    gameState = new GameStateHandler()
    controller = new Controller()
    fun run()
        while (running)
            processInput()
            updateGame()
            refresh()
    end run
end GameEngine
        \end{verbatim}
    \end{minipage}
\end{center}

\section{Model}
Le \textbf{Model} represente le noyau de la logic de jeu, pour que le controlleur ai acces au composant du model, on utilise une class de type facade StateManager.

\subsection{Level}
Level, class de type container, contient tout les elements de jeu (Palette, Brick, TextContainer, etc).
Methode all(), renvoie un std::tuple contenant tout les elements du level.
Utiliser std::apply pour iterer sur le tuple.
\url{https://stackoverflow.com/questions/1198260/how-can-you-iterate-over-the-elements-of-an-stdtuple}
\subsection{StateManager}
Contient les mothodes pour modifier les composant de level, contient un pointeur vers level.
\begin{center}
    \begin{minipage}[c]{0.5\textwidth}
        \begin{verbatim}
class Paddle
    int pos
    moveLeft(dx, dy)
end class
class StateManager
    movePaddleRight()
    ...
end class
class Level
    Paddle
    ...
end class
        \end{verbatim}
    \end{minipage}
\end{center}

\section{Controlleur}
Contient un StateManageur, Level et Renderer, recupere le input utilisateur, modifie les objets de jeu et render le resultat a chaque frame

\begin{center}
    \begin{minipage}[c]{0.5\textwidth}
        \begin{verbatim}
class Controller {
    int getInput() {
        if (keyPressed(LEFT)) return LEFT;
        if (keyPressed(RIGHT)) return RIGHT;
        \end{verbatim}
    \end{minipage}
\end{center}

\section{Inheritance}
Inheritance is a core concept in object-oriented programming where a class (child) derives from another class (parent) and inherits its properties and methods. This establishes an "is-a" relationship between the child and parent classes. However, inheritance can lead to tightly coupled code and deep hierarchies, making the system harder to maintain and extend.

\begin{center}
    \begin{minipage}[c]{0.5\textwidth}
        \begin{verbatim}
class Animal {
public:
    virtual void sound() = 0;
};

class Dog : public Animal {
public:
    void sound() override { std::cout << "Bark"; }
};
        \end{verbatim}
    \end{minipage}
\end{center}

\section{Composition}
Composition is a design principle where objects are composed of other objects to achieve complex functionality. This follows the "has-a" relationship, where one class contains another. Composition provides more flexibility, as components can be easily swapped or modified at runtime without modifying the entire class structure.

\begin{center}
    \begin{minipage}[c]{0.5\textwidth}
        \begin{verbatim}
class SoundBehavior {
public:
    virtual void sound() = 0;
};

class BarkBehavior : public SoundBehavior {
public:
    void sound() override { std::cout << "Bark"; }
};

class Dog {
private:
    SoundBehavior* soundBehavior;
public:
    Dog(SoundBehavior* sb) : soundBehavior(sb) {}
    void makeSound() { soundBehavior->sound(); }
};
        \end{verbatim}
    \end{minipage}
\end{center}

\end{document}
